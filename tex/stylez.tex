% All the packages
\usepackage{microtype}
\usepackage{url}
\usepackage{amsmath}
\usepackage{mathtools}
\usepackage{esint}
\usepackage{amssymb}
\usepackage{natbib}
\usepackage{multirow}
\usepackage{graphicx}
\usepackage{scalerel}
\usepackage{tikz}
\usepackage{calc}
\usepackage{etoolbox}
\usepackage{marginnote}
\usepackage{nicefrac}
\usepackage{tabstackengine}
\usepackage{diagbox}
\usepackage{xcolor}
\usepackage[makeroom]{cancel}
\usepackage{mathdots}
\usepackage{bbm}
\usepackage{booktabs}
\usepackage{xspace}
\usepackage{python}
\stackMath

% Bibliography stuff
\bibliographystyle{aasjournal}

% Shorthand for this paper
\newcommand{\starry}{\textsf{starry}\xspace}

% References to text content
\newcommand{\documentname}{\textsl{article}}
\newcommand{\figureref}[1]{\ref{fig:#1}}
\newcommand{\Figure}[1]{Figure~\figureref{#1}}
\newcommand{\figurelabel}[1]{\label{fig:#1}}
\renewcommand{\eqref}[1]{\ref{eq:#1}}
\newcommand{\Eq}[1]{Equation~(\eqref{#1})}
\newcommand{\eq}[1]{\Eq{#1}}
\newcommand{\eqalt}[1]{Equation~\eqref{#1}}
\newcommand{\eqlabel}[1]{\label{eq:#1}}

% Add script hyperlinks as margin notes
\definecolor{proofcolor}{rgb}{0.1216,0.4667,0.7059}
\newcommand{\script}[1]{\marginnote{\href{https://github.com/rodluger/starry/tree/master/tex/figures/#1.py}{\color{proofcolor}\scriptsize\textbf{[script]}}}}
\newcommand{\notebook}[1]{\marginnote{\href{https://github.com/rodluger/starry/tree/master/tex/notebooks/#1.nb}{\color{proofcolor}\scriptsize\textbf{[notebook]}}}}
\newcommand{\todo}[1]{\marginnote{\color{red}\textbf{todo:}\\\scriptsize\textbf{#1}}}

% Force margin notes to always be on the right side
% https://tex.stackexchange.com/a/69624
\makeatletter
\long\def\@mn@@@marginnote[#1]#2[#3]{%
  \begingroup
    \ifmmode\mn@strut\let\@tempa\mn@vadjust\else
      \if@inlabel\leavevmode\fi
      \ifhmode\mn@strut\let\@tempa\mn@vadjust\else\let\@tempa\mn@vlap\fi
    \fi
    \@tempa{%
      \vbox to\z@{%
        \vss
        \@mn@margintest
        \if@reversemargin\if@tempswa
            \@tempswafalse
          \else
            \@tempswatrue
        \fi\fi
          \rlap{%
            \ifx\@mn@currxpos\relax
              \kern\marginnoterightadjust
              \if@mn@verbose
                \PackageInfo{marginnote}{%
                  xpos not known,\MessageBreak
                  using \string\marginnoterightadjust}%
              \fi
            \else\ifx\@mn@currxpos\@empty
                \kern\marginnoterightadjust
                \if@mn@verbose
                  \PackageInfo{marginnote}{%
                    xpos not known,\MessageBreak
                    using \string\marginnoterightadjust}%
                \fi
              \else
                \if@mn@verbose
                  \PackageInfo{marginnote}{%
                    xpos seems to be \@mn@currxpos,\MessageBreak
                    \string\marginnoterightadjust
                    \space ignored}%
                \fi
                \begingroup
                  \setlength{\@tempdima}{\@mn@currxpos}%
                  \kern-\@tempdima
                  \if@twoside\ifodd\@mn@currpage\relax
                      \kern\oddsidemargin
                    \else
                      \kern\evensidemargin
                    \fi
                  \else
                    \kern\oddsidemargin
                  \fi
                  \kern 1in
                \endgroup
              \fi
            \fi
            \kern\marginnotetextwidth\kern\marginparsep
            \vbox to\z@{\kern\marginnotevadjust\kern #3
              \vbox to\z@{%
                \hsize\marginparwidth
                \linewidth\hsize
                \kern-\parskip
                \marginfont\raggedrightmarginnote\strut\hspace{\z@}%
                \ignorespaces#2\endgraf
                \vss}%
              \vss}%
          }%
      }%
    }%
  \endgroup
}
\makeatother

% Math stuff
\newcommand{\ii}{\ensuremath{\mathbf{i}}}
\newcommand{\T}{\ensuremath{\mathrm{T}}}
\newcommand{\dd}{\ensuremath{ \mathrm{d}}}
\newcommand{\unit}[1]{{\ensuremath{\mathrm{#1}}}}
\newcommand{\bvec}[1]{{\ensuremath{\mathbf{#1}}}}
\newcommand{\avec}[1]{{\ensuremath{\vec{\mathbf{#1}}}}}
\newcommand{\x}{\ensuremath{\mbox{$x$}}}
\newcommand{\y}{\ensuremath{\mbox{$y$}}}
\newcommand{\z}{\ensuremath{\mbox{$z$}}}
\newcommand{\xhat}{\ensuremath{\mathbf{\hat{x}}}}
\newcommand{\yhat}{\ensuremath{\mathbf{\hat{y}}}}
\newcommand{\zhat}{\ensuremath{\mathbf{\hat{z}}}}
\DeclareMathAlphabet\mathbfcal{OMS}{cmsy}{b}{n}
\DeclareMathOperator{\Tr}{Tr}
\DeclarePairedDelimiter\ceil{\lceil}{\rceil}
\DeclarePairedDelimiter\floor{\lfloor}{\rfloor}
\definecolor{dim}{rgb}{0.8,0.8,0.8}
\newcolumntype{L}[1]{>{\raggedright\let\newline\\\arraybackslash\hspace{0pt}}m{#1}}
\setcounter{MaxMatrixCols}{20}
\newcommand{\sinphi}{\ensuremath{\mbox{$u$}}}
\newcommand{\sinlambda}{\ensuremath{\mbox{$v$}}}
\newcommand{\bigdot}{\scaleto{\cdot}{6pt}}

% Bases
\newcommand{\pbasis}{\ensuremath{\bvec{\tilde{p}}}}
\newcommand{\gbasis}{\ensuremath{\bvec{\tilde{g}}}}
\newcommand{\ybasis}{\ensuremath{\bvec{\tilde{y}}}}
\newcommand{\pbasisn}{\ensuremath{\tilde{p}_n}}
\newcommand{\gbasisn}{\ensuremath{\tilde{g}_n}}
\newcommand{\ybasisn}{\ensuremath{\tilde{y}_n}}
\newcommand{\AOne}{\ensuremath{\bvec{A_1}}}
\newcommand{\ATwo}{\ensuremath{\bvec{A_2}}}

% Code examples
\usepackage{listings}
\definecolor{codegreen}{rgb}{0,0.6,0}
\definecolor{codegray}{rgb}{0.5,0.5,0.5}
\definecolor{codepurple}{rgb}{0.58,0,0.82}
\definecolor{backcolour}{rgb}{0.95,0.95,0.95}
\lstdefinestyle{mystyle}{
    backgroundcolor=\color{backcolour},
    commentstyle=\color{codegreen},
    keywordstyle=\color{magenta},
    numberstyle=\tiny\color{codegray},
    stringstyle=\color{codepurple},
    basicstyle=\small\ttfamily,
    breakatwhitespace=false,
    breaklines=true,
    captionpos=b,
    keepspaces=true,
    numbers=left,
    numbersep=5pt,
    showspaces=false,
    showstringspaces=false,
    showtabs=false,
    tabsize=2
}
\lstset{style=mystyle}

% Inverse diagonal dots
\makeatletter
\def\Ddots{\mathinner{\mkern1mu\raise\p@
\vbox{\kern7\p@\hbox{.}}\mkern2mu
\raise4\p@\hbox{.}\mkern2mu\raise7\p@\hbox{.}\mkern1mu}}
\makeatother

% Typography obsessions
\setlength{\parindent}{3.0ex}
\renewcommand\quad{\hskip\fontdimen3\font}
